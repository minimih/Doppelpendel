\documentclass[12pt]{article}

\usepackage[ngerman]{babel}
\usepackage[babel,german=swiss]{csquotes} % Schweizer

\usepackage{graphicx}
\usepackage{amsmath, amsthm, amssymb}
\usepackage{mathtools}
\usepackage{array}
\usepackage{fancyhdr}
\usepackage{xcolor}
\usepackage{wrapfig}
\usepackage{tabularx}
\usepackage{float}
\usepackage{natbib}
\usepackage{parskip}
\usepackage{tocloft}

\usepackage{fontspec,xltxtra,xunicode}
\defaultfontfeatures{Mapping=tex-text}
\setromanfont[Mapping=tex-text]{Calibri}
\setsansfont[Scale=MatchLowercase,Mapping=tex-text]{Gill Sans}
\setmonofont[Scale=MatchLowercase]{Andale Mono}
\linespread{1.2}

% Definition der Seite
\usepackage[paper=a4paper]{geometry} 

% Header Layout
\pagestyle{fancy}
\fancyhf{}
\fancyhead[L]{Projektarbeit: Doppelt inverses Pendel}
\fancyfoot[C]{\thepage}
\renewcommand{\headrulewidth}{0.1pt}
\renewcommand{\footrulewidth}{0.0pt}

\setlength{\headheight}{15pt}
\setcounter{secnumdepth}{4}
\setcounter{tocdepth}{4}

% PDF Einstellungen
\usepackage[draft=false, debug=false]{hyperref}
\hypersetup{
	pdfauthor={Michael Höhn, Stefan Hauenstein},
	pdftitle={Projektarbeit: Doppelt inverses Pendel},
	bookmarks,
	colorlinks,%
    citecolor=black,%
    filecolor=black,%
    linkcolor=black,%
    urlcolor=black
}

% Paragraph
\setlength\parindent{0pt} 
\setlength{\parskip}{0.75em}

% set graphics
\graphicspath{{./images/}}
\DeclareGraphicsExtensions{.png, .jpg}

% set equation numbering
\numberwithin{equation}{subsection}

\makeatletter
\renewcommand\paragraph{\@startsection{paragraph}{4}{\z@}%
  {-3.25ex\@plus -1ex \@minus -.2ex}%
  {1.5ex \@plus .2ex}%
  {\normalfont\normalsize\bfseries}}

% fig source
\newcommand{\figsourcefont}{\footnotesize}
\newcommand{\figsource}[1]{%
  \addtocontents{lof}{%
    {\leftskip\cftfigindent
     \advance\leftskip\cftfignumwidth
     \rightskip\@tocrmarg
     \figsourcefont#1\protect\par}%
  }%
 }

% No dots in TOC
\renewcommand{\@dotsep}{10000}
\makeatother

\title{\vspace{-1cm}\linespread{1}\begin{flushleft}\normalsize{Zürcher Hochschule für Angewandte Wissenschaften\\Studiengang Informatik\\}\end{flushleft}\vspace{2cm}\Large{Projektarbeit: Doppelt inverses Pendel}\\\vspace{2cm} \includegraphics[scale=1]{title.png}\\\tiny{Bild: scienceworld.wolfram }}

\author{Michael Höhn\\\href{mailto:hoehnmic@students.zhaw.ch}{hoehnmic@students.zhaw.ch} \and Stefan Hauenstein\\\href{mailto:hauenste@students.zhaw.ch}{hauenste@students.zhaw.ch}}

\date{\today}

% Titelseite
\begin{document}
\maketitle\thispagestyle{empty}

\newpage

\begin{abstract}
Bei dieser Projektarbeit handelt es sich um die programmiertechnische Umsetzung des Chaotischen Systems eines inversen Doppelpendels in ActionScript 3.
\end{abstract}
\newpage

% Inhaltsverzeichnis
\tableofcontents
\newpage

\section{Projektbeschreibung}
\subsection{Danksagung}
Wir möchten an dieser Stelle Herrn Georg Brügger und Herrn Lukas Eppler danken, ohne ihre wertvolle Unterstützung und Anregungen wäre dieses Software Projekt nur schwer umsetzbar gewesen.

\subsection{Aufgabenstellung durch den Experten}
Das Schwingverhalten eines idealen „Doppelt Inversen Pendel“ soll untersucht und durch geeignete Physikalische und Mathematische Beziehungen nach gebildet werden. Physikalische Effekte wie Reibung und Luftwiderstand werden vernachlässigt, Abmessungen und Materialien müssen als Eingabegrössen variabel gestaltet werden.

Der Vorgang wird in „Echtzeit“ am Bildschirm des PC's Graphisch dargestellt visualisiert.

Zusätzliche Angaben des Verhaltens können auf der gleichen Bildschirmseite eingeblendet werden.

\subsection{Motivation}
Physikalische Eigenschaften mit einer Objekt Orientierten Programmiersprache umzusetzen reizte uns schon immer. Die  Aufgabenstellung ermöglichte uns das theoretisch gelernte aus dem Studienfach Numerik in Verbindung mit realen Physikalischen Grundlagen in einer praktischen Anwendung umzusetzen.

Auch ermöglichte es die algebraischen Zusammenhänge aus vergangenen Vorlesungen besser zu verstehen und Lücken aufzufüllen.

\subsection{Iterationsplan}
\newcolumntype{L}{>{\raggedright\arraybackslash}X}
\begin{tabularx}{\textwidth}{|l|l|L|}
	\hline
	\textbf{Iteration} & \textbf{Termin}     & \textbf{Task}\\
	\hline
	1                  & Mi 04.04.2012       & \begin{itemize}
													\item Physikalische Eigenschaften des Doppelpendels aufstellen
													\item Evaluierung der Programmiersprache
													\item Gleichungssystem aufstellen
											   \end{itemize}\\
	\hline
	2                  & Fr 27.04.2012       & \begin{itemize}
													\item Grafische Benutzeroberfläche
													\item Programmierung des ersten Pendels
													\item Programmierung des zweiten Pendels
													\item Runge Kutta definieren
											   \end{itemize}\\
	\hline
	3                  & Mi 23.05.2012       & \begin{itemize}
													\item Pendel einstellen
													\item Projektdokumentation
													\item Präsentation
											   \end{itemize}\\
	\hline
	4                  & Präsentationstermin & \begin{itemize}
													\item Präsentation
													\item Projektdokumentation
											   \end{itemize}\\
	\hline
\end{tabularx}

\subsection{Beteiligte Personen}
\begin{tabularx}{\textwidth}{|X|X|}
	\hline
	\textbf{Funktion} & \textbf{Name} \\
	\hline
	Projekt Team      & Stefan Hauenstein \\
	                  & Michael Höhn \\
	\hline
	Experte           & Georg Brügger\\
	\hline
	Scrum Master      & Lukas Eppler\\
	\hline
\end{tabularx}

\newpage
\section{Anforderungsanalyse}
\subsection{Zielbestimmung}
Es soll ein Programm entwickelt werden das die physikalischen und chaotischen Eigenschaften des inversen Doppelpendels grafisch darstellt.

Die einzelnen Segmente des Pendels sollen durch Benutzereingaben über Tastatur und/oder Maus einstellbar sein.

\subsection{Zielbestimmung}
\begin{itemize}
	\item Die Umgebung des Doppelpendels soll als ideal gelten.
	\item Das Ausgangsmaterial wird als Aluminium Stab definiert (Dichte $2,7 [g/cm^3]$)
	\item Die Gravitation wird mit $9.81 [m/s^2]$ definiert
	\item Wegen der chaotischen Eigenschaft des Pendels können keine aussagenkräftigen Tests implementiert werden.
\end{itemize}

\subsection{Produktfunktionen}
\begin{itemize}
	\item Der Benutzer kann eine Konfiguration als XML-File mit der Erweiterung DIP einlesen.
	\item Der Benutzer kann die einzelnen Glieder des Pendels mit der Maus positionieren.
	\item Während das Pendel in Bewegung ist werden Winkel und Geschwindigkeit angezeigt.
	\item Der Benutzer kann den Ablauf starten, jederzeit stoppen und zurückstellen.
\end{itemize}

\newpage
\section{Umsetzung}

\begin{figure}[h]
	\centering
	\includegraphics[width=\textwidth]{function.png}
	\caption{Funktionsweise des Doppelpendels}
	\label{fig:function}
\end{figure}

\subsection{Verwendete Technologien}

\subsubsection{ActionScript 3}
Das Programm wird mittels der Objektorientierten Sprache ActionScript 3 (AS3) und dem Flex SDK in der Eclipse Entwicklungsumgebung Programmiert. Das Programm ist dank dem Flashplayer auf fast jedem Betriebssystem und Browser lauffähig.
Die Implementierung mittels Flex SDK und AS3 ist speziell empfehlenswert, da das Flex Framework speziell für die Animation ausgelegt ist und es dadurch zu einer grossen Zeiteinsparung bei der Entwicklung kommt.

\subsubsection{Nummerisches Verfahren}
Um die Fehlerkomulierten Abweichungen der Berechnung auf möglichst lange Zeit klein zu halten wird das Numerische Verfahren von Runge-Kutta in der 4. Ordnung verwendet. Der Rechenaufwand für dieses Verfahren ist höher als bei anderen Verfahren wie z.B. dem Euler-Verfahren. Seine Genauigkeit auf grosse Zeitintervalle gleichen diesen Mehraufwand wieder aus.

\paragraph{Runge-Kutta Verfahren 4. Ordnung}
Das klassische Runge-Kutta-Verfahren (nach Carl Runge und Martin Wilhelm Kutta) ist ein spezielles explizites 4-stufiges Runge-Kutta-Verfahren zur numerischen Lösung von Anfangswertproblemen. Runge hat als erster (1895) ein mehrstufiges Verfahren angegeben und Kutta die allgemeine Form expliziter s-stufiger Verfahren.

Das klassische Runge-Kutta-Verfahren verwendet den Ansatz, Ableitungen durch Differenzenquotienten zu approximieren. Die dabei bei nichtlinearen Funktionen notwendigerweise auftretenden können durch geeignete Kombinationen verschiedener Differenzquotienten reduziert werden. Das Runge-Kutta-Verfahren 4. Ordnung ist eine solche Kombination, die Diskretisierungsfehler bis zur dritten Ableitung kompensiert.

\begin{align*}
	y(t)' &= f(t,y(t)) \quad \text{mit} \quad y(0) = y_0
	\\
	\\
	y^{k+1} &= y^k + h \left(\frac{1}{6}k_1 + \frac{2}{6}k_2 + \frac{2}{6}k_3 + \frac{1}{6}k_4 \right)
	\\
	\\
	\text{mit} \quad k_1 &= f(t_k, y^k)\\
	k_2 &= f\left(t_k + \frac{h}{2}, y^k + \frac{h}{2}k_1 \right)\\
	k_3 &= f\left(t_k + \frac{h}{2}, y^k + \frac{h}{2}k_2 \right)\\
	k_4 &= f\left(t_k + h, y^k + h k_3 \right)
\end{align*}

\newpage
\section{Physikalische und Mathematische Zusammenhänge}
\subsection{Physikalische Zusammenhänge}
Um die Gleichungen des Ausgangsproblems zu definieren, muss man sich erst mit den physikalischen Eigenschaften eines Pendels beschäftigen. Kräfte welche aus der Bewegung entstehen  kombinieren sich mit Statischen. Kräfte des ersten Pendels beeinflussen die des Zweiten. Im Folgenden werden die Kinematik und die Kräfte des Doppelpendels zum gesuchten Ausgansproblem kombiniert.

\subsubsection{Kinematik des Doppelpendels}
\begin{figure}[!h]
	\begin{minipage}[!b]{0.4\textwidth}
		\centering
		\includegraphics[scale=1]{dbl_pendulum.png}
		\caption{Doppelpendel}
		\figsource{Quelle: www.myphysicslab.com}
		\label{fig:energy}
	\end{minipage}
	\begin{minipage}[!t]{\textwidth}
		\vspace{0pt}\raggedright
		$x$ = Horizontale Position des Massenschwerpunkts\\
		$y$ = Vertikale Position des Massenschwerpunkts\\
		$\theta$ = Winkel\\
		$L$ = Pendellänge
	\end{minipage}
\end{figure}

Ausgangslage für die Berechnung ist das obere Pendel. Das erste Pendel wird als Subskript $1$ definiert mit $x_1$ und $y_1$. Äquivalent dazu wird das zweite Pendel als Subskript $2$ definiert mit $x_2$ und $y_2$.

Als erstes werden die trigonometrischen Gleichungen aufgestellt:

\begin{equation} \label{eq:energy:1_1}
	x_1 = L_1\sin\theta_1
\end{equation}
\begin{equation} \label{eq:energy:1_2}
	y_1 = -L_1\cos\theta_1
\end{equation}

\begin{equation} \label{eq:energy:2_1}
	x_2 = x_1 + L_2\sin\theta_2
\end{equation}
\begin{equation} \label{eq:energy:2_2}
	y_2 = y_1 - L_2\cos\theta_2
\end{equation}

Danach wird aus den Gleichungen (\ref{eq:energy:1_1} - \ref{eq:energy:2_2}) die erste und zweite Ableitung gebildet.

Erste Ableitung (Geschwindigkeit):
\begin{equation} \label{eq:velocity:1_1}
	x_1' = \theta_1' L_1\cos\theta_1
\end{equation}
\begin{equation} \label{eq:velocity:1_2}
	y_1' = \theta_1' L_1\sin\theta_1
\end{equation}

\begin{equation} \label{eq:velocity:2_1}
	x_2' = x_1' + \theta_2' L_2\cos\theta_2
\end{equation}
\begin{equation} \label{eq:velocity:2_2}
	y_2' = y_1' + \theta_2' L_2\sin\theta_2
\end{equation}

Zweite Ableitung (Beschleunigung):
\begin{equation} \label{eq:acceleration:1_1}
	x_1'' = -{\theta_1'}^2 L_1 \sin\theta_1 + \theta_1'' L_1 \cos\theta_1
\end{equation}
\begin{equation} \label{eq:acceleration:1_2}
	y_1'' = {\theta_1'}^2 L_1 \cos\theta_1 + \theta_1'' L_1 \sin\theta_1
\end{equation}

\begin{equation} \label{eq:acceleration:2_1}
	x_2'' = x_1'' - {\theta_2'}^2 L_2\sin\theta_2 + \theta_2'' L_2 \cos\theta_2
\end{equation}
\begin{equation} \label{eq:acceleration:2_2}
	y_2'' = y_1'' + {\theta_2'}^2 L_2\cos\theta_2 + \theta_2'' L_2 \sin\theta_2
\end{equation}

\subsubsection{Kräfte im Doppelpendel}
Die Kräfte werden pro Pendel berechnet und in einzelne Gleichungen gesetzt. Auf Grundlage des zweiten newtonschen Gesetzes $(F = m \cdot a)$ können folgende Gleichungen für das Doppelpendel erstellt werden:

\paragraph{Oberes Pendel}
\begin{figure}[H]
	\begin{minipage}[!b]{0.5\textwidth}
		\centering
		\includegraphics[scale=1]{dbl_pendulum_m1.png}
		\caption{Kräfte im oberen Pendel\cite{pl_pendulum}}
		\figsource{Quelle: www.myphysicslab.com}
		\label{fig:force_1}
	\end{minipage}
	\begin{minipage}[!t]{\textwidth}
		\vspace{0pt}\raggedright
		$T_1$ = Kraft im oberen Pendel\\
		$m_1$ = Masse des oberen Pendels\\
		$g$ = Gravitations Konstante
	\end{minipage}
\end{figure}

Anhand von Abbildung \ref{fig:force_1} können die folgenden Gleichungen aufgestellt werden.
\begin{equation} \label{eq:force:1_1}
	m_1 x_1'' = -T_1\sin\theta_1 + T_2\sin\theta_2
\end{equation}
\begin{equation} \label{eq:force:1_2}
	m_1 y_1'' = T_1\cos\theta_1 + T_1\cos\theta_2 - m_1 g
\end{equation}

\paragraph{Unteres Pendel}
\begin{figure}[H]
	\begin{minipage}[!b]{0.5\textwidth}
		\centering
		\includegraphics[scale=1]{dbl_pendulum_m2.png}
		\caption{Kräfte im unteren Pendel}
		\figsource{Quelle: www.myphysicslab.com}
		\label{fig:force_2}
	\end{minipage}
	\begin{minipage}[!t]{\textwidth}
		\vspace{0pt}\raggedright
		$T_2$ = Kraft im unteren Pendel\\
		$m_2$ = Masse des unteren Pendels\\
		$g$ = Gravitations Konstante
	\end{minipage}
\end{figure}

Anhand von Abbildung \ref{fig:force_2} können die folgenden Gleichungen aufgestellt werden.
\begin{equation} \label{eq:force:2_1}
	m_2 x_2'' = -T_2\sin\theta_2
\end{equation}
\begin{equation} \label{eq:force:2_2}
	m_2 y_2'' = T_2\cos\theta_2 - m_2 g
\end{equation}

\subsubsection{Umformen der Gleichungen}
Um zu der endgültigen Differenziellen Gleichung für die Berechnung zu gelangen müssen die einzelnen Gleichungen algebraisch umgeformt werden.
Angefangen bei den Kräftegleichungen. Gleichung (\ref{eq:force:2_1}) und (\ref{eq:force:2_2}) werden nach $T_2\sin\theta_2$ bzw. nach $T_2\cos\theta_2$ aufgelöst und in die Gleichungen (\ref{eq:force:1_1}) und (\ref{eq:force:1_2}) eingesetzt.
\begin{equation} \label{eq:dForce:1}
	m_1 x_1'' = -T_1\sin\theta_1 - m_2 x_2''
\end{equation}
\begin{equation} \label{eq:dForce:2}
	m_1 y_1'' = T_1\cos\theta_1 -m_2 y_2'' - m_2 g - m_1 g
\end{equation}

Gleichung (\ref{eq:dForce:1}) wird mit $\cos\theta_1$ und Gleichung (\ref{eq:dForce:2}) mit $\sin\theta_1$ erweitert und nach $T_1\sin\theta_1\cos\theta_1$ aufgelöst.
\begin{equation} \label{eq:thetaForce:1_1}
	T_1\sin\theta_1\cos\theta_1 = -\cos\theta_1(m_1 x_1'' + m_2 x_2'')
\end{equation}
\begin{equation} \label{eq:thetaForce:1_2}
	T_1\sin\theta_1\cos\theta_1 = \sin\theta_1(m_1 y_1'' + m_2 y_2'' + m_2 g + m_1 g)
\end{equation}

Daraus ergibt sich die erste Teil der Bewegungsgleichung.
\begin{equation} \label{eq:motion:1}
	\sin\theta_1(m_1 y_1'' + m_2 y_2'' + m_2 g + m_1 g) = -\cos\theta_1(m_1 x_1'' + m_2 x_2'')
\end{equation}

Um den zweiten Teil der Gleichung zu erhalten wird die Gleichung (\ref{eq:force:2_1}) mit $\cos\theta_2$ und die (\ref{eq:force:2_2}) mit $\sin\theta_2$ erweitert.
\begin{equation} \label{eq:thetaForce:2_1}
	T_2\sin\theta_2\cos\theta_2 = -\cos\theta_2(m_2 x_2'')
\end{equation}
\begin{equation} \label{eq:thetaForce:2_2}
	T_2\sin\theta_2\cos\theta_2 = \sin\theta_2(m_2 y_2'' + m_2 g)
\end{equation}

Der zweite Teil der Gleichung ergibt somit
\begin{equation} \label{eq:motion:2}
	\sin\theta_2(m_2 y_2'' + m_2 g) = -\cos\theta_2(m_2 x_2'')
\end{equation}

$x_2''$, $x_2''$, $y_1''$ und $y_2''$ werden durch die Gleichungen (\ref{eq:acceleration:1_1} – \ref{eq:acceleration:2_2}) ersetzt und mit Hilfe des Lagrange-Formalismus nach $\theta_1''$ und $\theta_2''$ aufgelöst.

Die endgültige Bewegungsgleichung für das Doppelpendel lautet somit:
\begin{equation} \label{eq:final:1}
	\theta_1'' = \frac{-g(2m_1+m_2)\sin\theta_1 - m_2 g \sin(\theta_1 - 2\theta_2) -2\sin(\theta_1 - \theta_2)m_2({\theta_2'}^2 L_2 + {\theta_1'}^2 L_1\cos(\theta_1 - \theta_2))}
	{L_1(2m_1 + m_2 - m_2 \cos(2\theta_1 - 2\theta_2))}
\end{equation}

\begin{equation} \label{eq:final:2}
	\theta_2'' = \frac{2\sin(\theta_1 - \theta_2)({\theta_1'}^2 L_1(m_1 + m_2) + g(m_1 + m_2)\cos\theta_1 + {\theta_2'}^2 L_2 m_2 \cos(\theta_1 - \theta_2))}
	{L_2 (2m_1 + m_2 - m_2 \cos(2 \theta_1 - 2\theta_2))}
\end{equation}

\subsection{Numerische Differentialgleichung}
Die Bewegungsgleichung (\ref{eq:final:1} und \ref{eq:final:2}) lässt sich aber noch nicht mittels des Runge-Kutta Algorithmus numerisch annähern. Um dies zu ermöglichen muss bedacht werden, dass die erste Ableitung von $\theta$ die Winkelgeschwindigkeit $\omega$ ist.

Daraus ergibt sich:
\begin{equation*}
	\theta_1' = \omega_1 \quad \text{und} \quad \theta_2' = \omega_2
\end{equation*}

damit erhält man für den Runge-Kutta Algorithmus folgende Schlussgleichungen:
\begin{equation} \label{eq:rk:1}
	\theta_1' = \omega_1
\end{equation}
\begin{equation} \label{eq:rk:2}
	\theta_2' = \omega_2
\end{equation}

\begin{equation} \label{eq:rk:3}
	\omega_1' = \frac{-g(2m_1+m_2)\sin\theta_1 - m_2 g \sin(\theta_1 - 2\theta_2) -2\sin(\theta_1 - \theta_2)m_2(\omega_2^2 L_2 + \omega_1^2 L_1\cos(\theta_1 - \theta_2))}
	{L_1(2m_1 + m_2 - m_2 \cos(2\theta_1 - 2\theta_2))}
\end{equation}

\begin{equation} \label{eq:rk:4}
	\omega_2' = \frac{2\sin(\theta_1 - \theta_2)(\omega_1^2 L_1(m_1 + m_2) + g(m_1 + m_2)\cos\theta_1 + \omega_2^2 L_2 m_2 \cos(\theta_1 - \theta_2))}
	{L_2 (2m_1 + m_2 - m_2 \cos(2 \theta_1 - 2\theta_2))}
\end{equation}

\newpage
\section{Software}
\subsection{Grafische Oberfläche}
Die Grafische Oberfläche wird als Skalierbares Modell aufgebaut. Der Vorteil in diesem Aufbau besteht darin, dass die Applikation auf jeder Bildschirmauflösung optimal dargestellt wird. Die Simulation des Pendels passt sich anhand der Fenstergrösse automatisch an.

\newpage
\bibliographystyle{plainnat}
\bibliography{books}

\newpage
\listoffigures
\listoftables

\end{document}  